\documentclass{seto}
\title{Desargues and his meme }
\author{Neal + Krishna (feat. Tiger)}
\date{13 June 2023, MOP}
\displaydate{MOP 2023}
\begin{document}
\maketitle
\begin{remark}[Note]
    This handout originally consisted of just the problem set.
    However, at G$2$ $2024$, I intended to talk about Desargue involution (but 
    didn't do this); as a result, motivation has been added.\\[4pt]
    In terms of problems, I retroactively have edited to include 
    even more recent instances of it. Some are more instructive than others\dots -n
\end{remark}
\toc
\section{Acknowledgement}
Eric Shen for teaching me this black magic and its very interesting
applications. Thanks so much Eric! -- Neal \\[4pt]
We also stole a problem from Tiger Zhang for the lecture, which I looked at a few minutes before lecturing:
\begin{block}[Problem (OMMC Main 2023)]
Define acute $\triangle ABC$ with circumcenter $O$. The circumcircle of
$\triangle ABO$ meets segment $BC$ at $D \ne B$, segment $AC$ at $F \ne A$, and
the Euler line of $\triangle ABC$ at $P \ne O$. The circumcircle of $\triangle
ACO$ meets segment $BC$ at $E \ne C$. Let $\overline{BC}$ and $\overline{FP}$
intersect at $X$, with $C$ between $B$ and $X$. If $BD=13$, $EC=8$, and $CX=27$,
find $DE$.
\end{block}

\section{Motivation}
As with projective in general, I want to justify its use.
In particular, though, there are people 
out there who consider the use of (D)DIT as `cheating'.\\
I believe (D)DIT has a hideous equivalent in length chasing as well.
But first and foremost, one uses it to access otherwise inaccessible intersections.
\begin{block}[Example: IGO Intermediate 2018/5]
Suppose that $ABCD$ is a parallelogram such that $\angle
DAC = 90^\circ$. Let $H$ be the foot of perpendicular from $A$ to $DC$, also let $P$
be a point along the line $AC$ such that the line $PD$ is tangent to the
circumcircle of the triangle $ABD$. Prove that $\angle PBA = \angle DBH$. 
\end{block}
Here, the point $H$ would otherwise be inaccessible without something like involution.
Actually, the difficulty of the proof of the theorem 
is why successful application to a problem leads to
a lot of progress if not obliterating the problem outright. 
As nukes are rather few and far between in geometry, they are sometimes considered cheating,
especially this one.
\section{Theory}
\begin{thm}[Definition: involution]
    An involution in an FE context is a function
    $f$ such that $f(f(x))=x$ for all $x$. \\
    In geometry, we expand this definition by requiring involutions to preserve
    cross ratio: 
    \[(WX;YZ)=(f(W),f(X);f(Y),f(Z)).\]
\end{thm}
Some important facts:
\begin{itemize}
\item By algebra, all involutions on a line (domain and range being the set of points on the line) are 
inversions: those of the form $x\to (ax+b)/(cx+d)$ for some constants $a,b,c,d$.
\item Involutions are determined by two pairs.
\item As a consequence of preserving cross ratio, we can project an involution 
of lines onto another line, to obtain an involution of points, and vice versa.
\item We don't know much about involutions on the set of lines throug a point, besides
the fact that rotations of $90^\circ$ and reflections in a fixed line (e.g. taking isogonals)
are involutions.
\end{itemize}
\begin{thm}[Desargues's involution theorem aka DIT]
Let points $A,B,C,D$ lie on conic $\gamma$, and consider any line $\ell$.
Then there exists an involution
\[(\ell\cap\gamma;\ell\cap\ol{AB},\ell\cap\ol{CD};
\ell\cap\ol{AC},\ell\cap\ol{BD}; \ell\cap\ol{AD},\ell\cap\ol{BC}).\]
\end{thm}
Application of DIT is usually not very hard to see- I see intersections of a random
line with a quadrilateral, I figure out the DIT application.
\begin{block}[Textbook example: CGMO 2017/7]
Let $ABCD$ be a cyclic quadrilateral with circumcircle $\omega_1$. Lines $AC$
and $BD$ intersect at point $E$, and lines $AD$,$BC$ intersect at point
$F$. Circle $\omega_2$ is tangent to segments $EB,EC$ at points $M,N$
respectively, and intersects with circle $\omega_1$ at points $Q,R$. Lines $BC,AD$
intersect line $MN$ at $S,T$ respectively. Show that $Q,R,S,T$ are concyclic.
\end{block}
\begin{thm}[Dual of DIT aka DDIT]
Suppose $ABCD$ has inconic $\omega$ and $E=\ol{AB}\cap\ol{CD}$ and $F=\ol{AD}\cap\ol{BC}$.
Let the tangents from any point $P$ to the conic be $\ell_1$ and $\ell_2$.\\
Then there exists the following involution:
\[(\ell_1,\ell_2;\ol{PA},\ol{PC};\ol{PB},\ol{PD};\ol{PE},\ol{PF}).\]
\end{thm}
The use of this theorem is much more varied than its dual. 

The theorem is less scary than it looks, because the desired involution
is usually something we can recognize.
\begin{block}[Example (Krishna Pothapragada)]
Let $a$, $b$, $c$, $d$ be complex numbers with magnitude $1$.
Let $z$ be a complex number such that 
\[\frac{z-2ab/(a+b)}{z-2cd/(c+d)}\text{ and } \frac{z-2ad/(a+d)}{z-2bc/(b+c)}\]
are purely imaginary. Prove that $\abs z=\sqrt 2$.
\end{block}
\begin{remark} To this day I do not know the proof of these theorems.\end{remark}
\section{Problems}
Approximately increasing difficulty....
\exercise[MPO 2024/1] Let $ABC$ be an acute scalene triangle 
and let $D$ be a point in its 
interior such that $\angle DBA = \angle DCA$. 
Points $E$ and $F$ lie inside segments $AC$ and $AB$ such that
$DE = DC$ and $DB = DF$. 
Is it possible that lines $AD$, $BE$, $CF$ are concurrent?
\exercise[APMO 2024/1] Let $ABC$ be an acute triangle. Let $D,E$ be respective
points on $\ol{AB},\ol{AC}$ such $BC\parallel\ol{DE}$.
Let $X$ be an interior point of $BCED$. Suppose rays $DX$ and $EX$
meet side $BC$ at points $P$ and $Q$, respectively, such that both $P$ and $Q$
lie between $B$ and $C$. Suppose that the circumcircles of triangles $BQX$ and
$CPX$ intersect at a point $Y \neq X$. Prove that the points $A, X$, and $Y$ are
collinear.
\begin{remark} balls\end{remark} $ $
\exercise[USA TST 2004/4] Let $ABC$ be a triangle. Choose a point $D$ in its
interior. Let $\omega_1$ be a circle passing through $B$ and $D$ and $\omega_2$
be a circle passing through $C$ and $D$ so that the other point of intersection
of the two circles lies on $AD$. Let $\omega_1$ and $\omega_2$ intersect side
$BC$ at $E$ and $F$, respectively. Denote by $X$ the intersection of $DF$, $AB$
and $Y$ the intersection of $DE, AC$. Show that $XY \parallel BC$.
\exercise[CJMO 2021/1] Let $ABC$ be an acute triangle, and let the feet of the
altitudes from $A$, $B$, $C$ to $\overline{BC}$, $\ol{CA}$, $\ol{AB}$ be $D$,
$E$, $F$, respectively. Points $X\neq F$ and $Y\neq E$ lie on lines $CF$ and
$BE$ respectively such that $\angle XAD = \angle DAB$ and $\angle YAD = \angle
DAC$. Prove that $X$, $D$, $Y$ are collinear.
\exercise[Serbia 2017/6] Let $k$ be the circumcircle of $\triangle ABC$ and let
$k_a$ be A-excircle .Let the two common tangents of $k,k_a$ cut $BC$ in
$P,Q$.Prove that $\angle PAB=\angle CAQ$.
\exercise[Taiwan TST 2014/3/3]Let $ABC$ be a triangle with circumcircle $\Gamma$
and let $M$ be an arbitrary point on $\Gamma$. Suppose the tangents from $M$ to
the incircle of $\triangle ABC$ intersect $\ol{BC}$ at two distinct points $X_1$
and $X_2$. Prove that the circumcircle of triangle $MX_1X_2$ passes through the
tangency point of the $A$-mixtilinear incircle with $\Gamma$.
\exercise[USA TST 2018/5 (by Evan)] Let $ABCD$ be a convex cyclic quadrilateral
which is not a kite, but whose diagonals are perpendicular and meet at $H$.
Denote by $M$ and $N$ the midpoints of $\overline{BC}$ and $\overline{CD}$. Rays
$MH$ and $NH$ meet $\overline{AD}$ and $\overline{AB}$ at $S$ and $T$,
respectively. Prove that there exists a point $E$, lying outside quadrilateral
$ABCD$, such that
\begin{itemize}
\item ray $EH$ bisects both angles $\angle BES$, $\angle TED$, and
\item $\angle BEN = \angle MED$.
\end{itemize}
\exercise[IMO 2019/2] In triangle $ABC$, point $A_1$ lies on side $BC$ and point
$B_1$ lies on side $AC$. Let $P$ and $Q$ be points on segments $AA_1$ and
$BB_1$, respectively, such that $PQ$ is parallel to $AB$. Let $P_1$ be a point
on line $PB_1$, such that $B_1$ lies strictly between $P$ and $P_1$, and $\angle
PP_1C=\angle BAC$. Similarly, let $Q_1$ be the point on line $QA_1$, such that
$A_1$ lies strictly between $Q$ and $Q_1$, and $\angle CQ_1Q=\angle CBA$.
\\[4pt]
Prove that points $P,Q,P_1$, and $Q_1$ are concyclic. 
\exercise[AoPS] Triangle $ABC$ has incircle $\omega$ and circumcircle $\Omega$,
and the former touches the sides at $D,E,F$ in order.  Let $\overline{EF}$ meet
$\Omega$ at $M_1,M_2$ and $\overline{DM_n}$ meet $\omega$ at $K_n$ for $n=1,2$.
Finally, let the tangents to $\Omega$ at $B,C$ meet at $T$. Then the lines
$TK_n$ touch $\omega$.
\exercise[MOP HW \#21] In acute scalene $\triangle ABC$ with circumcenter $O$,
orthocenter $H$, Kosnita point $X_{54}=K$, define $P=(HO)\cap(BOC)$, $Q$ be the
foot from line onto $AO$. Prove that $P,Q,K$ are collinear. (The Kosnita point
is the point at which the line through $A$ and the circumcenter of $\triangle
BOC$ and the other two analogous lines concur; it is the isogonal conjugate of
the nine-point center. )
\exercise[Shortlist 2012/G8] Let $ABC$ be a triangle with circumcircle $\omega$
and $\ell$ a line without common points with $\omega$. Denote by $P$ the foot of
the perpendicular from the center of $\omega$ to $\ell$. The side-lines
$BC,CA,AB$ intersect $\ell$ at the points $X,Y,Z$ different from $P$. Prove that
the circumcircles of the triangles $AXP$, $BYP$ and $CZP$ have a common point
different from $P$ or are mutually tangent at $P$.
\exercise[Shortlist 2021/G8] Let $ABC$ be a triangle with circumcircle $\omega$
and let $\Omega_A$ be the $A$-excircle. Let $X$ and $Y$ be the intersection
points of $\omega$ and $\Omega_A$. Let $P$ and $Q$ be the projections of $A$
onto the tangent lines to $\Omega_A$ at $X$ and $Y$ respectively. The tangent
line at $P$ to the circumcircle of the triangle $APX$ intersects the tangent
line at $Q$ to the circumcircle of the triangle $AQY$ at a point $R$. Prove that
$\overline{AR} \perp \overline{BC}$.
\exercise[USAMO 2012/5] Let $P$ be a point in the plane of $\triangle ABC$, and
$\gamma$ a line passing through $P$. Let $A', B', C'$ be the points where the
reflections of lines $PA, PB, PC$ with respect to $\gamma$ intersect lines $BC,
AC, AB$ respectively. Prove that $A', B', C'$ are collinear.
\exercise[Shortlist 2022/G8] Let $AA'BCC'B'$ be a convex cyclic hexagon such
that $AC$ is tangent to the incircle of the triangle $A'B'C'$, and $A'C'$ is
tangent to the incircle of the triangle $ABC$. Let the lines $AB$ and $A'B'$
meet at $X$ and let the lines $BC$ and $B'C'$ meet at $Y$. \\[3pt]
Prove that if $XBYB'$ is a convex quadrilateral, then it has an incircle.
\exercise[China 2020/2] Let ABC be a triangle, and let the bisector of $\angle
A$ intersect $\ol{BC}$ at $D$. Point $P$ lies on line $AD$ such that $P, A, D$
are collinear in that order. Suppose $\ol{PQ}$ is tangent to $(ABD)$ at $Q$,
$\ol{PR}$ is tangent to $(ACD)$ at $R$, and $Q$ and $R$ lie on opposite sides of
line $AD$. Let $K = BR \cap CQ$. Prove that if the line through $K$ parallel to
$BC$ intersects lines $QD, AD, RD$ at $E, L, F$ , respectively, then $EL = KF$ .
\exercise[TSTST 2024/8] Let $ABC$ be a scalene triangle, and let $D$ be a point
on side $BC$ satisfying $\angle BAD=\angle DAC$. Suppose that $X$ and $Y$ are
points inside $ABC$ such that triangles $ABX$ and $ACY$ are similar and
quadrilaterals $ACDX$ and $ABDY$ are cyclic. Let lines $BX$ and $CY$ meet at $S$
and lines $BY$ and $CX$ meet at $T$. Prove that lines $DS$ and $AT$ are
parallel.
\exercise[\href{https://aops.com/community/p30812430}{aops.com/community/p30812430}] 
Let $I$ be the incenter of $\triangle ABC$. The tangent
line of the incircle which is parallel to $BC$ intersects $AC,AB$ at points
$E,F$, respectively. Points $P,Q$ are on $CF$ such that $IP\perp IQ$. Line $BP$
and $EQ$ meet at $K$. Point $L$ is on $CF$ such that $IL\perp IK$.
Prove that $KL$ is tangent to the incircle.
\exercise[TSTST 2023/6] Let $ABC$ be a scalene triangle and let $P$ and $Q$ be two distinct points in its interior. Suppose that the angle bisectors of $\angle PAQ$,$\angle PBQ,$ and $\angle PCQ$ are the altitudes of triangle $ABC$. Prove that the midpoint of $\overline{PQ}$ lies on the Euler line of $ABC$. 
(the author was splashed again :skull:)
\begin{remark}
    I am not sure if the last one can actually be DDIT'd.
\end{remark}$ $
\exercise[ELMO Shortlist 2023?] to be added later when i remember, skibidi xooks rbo 
\exercise[ELMO SL 2024/G8] Let $ABC$ be a triangle, 
and let $D$ be a point on the internal angle bisector of $BAC$. 
Let $x$ be the ellipse with foci $B$ and $C$ passing through $D$, 
$y$ be the ellipse with foci $A$ and $C$ passing through $D$, 
and $z$ be the ellipse with foci $A$ and $B$ passing through $D$. 
Ellipses $x$ and $z$ intersect at distinct points $D$ and $E$, 
and ellipses $x$ and $y$ intersect at distinct points $D$ and $F$. 
Prove that $AD$ bisects angle $EAF$.
\end{document}
